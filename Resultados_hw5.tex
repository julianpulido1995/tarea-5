\documentclass[12pt]{article}
\usepackage{amsmath,amssymb,graphicx}
\usepackage[margin=2cm]{geometry}
\author{Julian Pulido}
\date{25 de Mayo de 2017}
\title{Tarea 5 de metodos computacionales}

\begin{document}
\maketitle
\begin{abstract}
En el presente documento se presentan los resultados de los dos puntos de la tarea 5 de metodos computacionales. El documento se compone de dos secciones para ambos puntos, cada una con sus respectivas graficas.La seccion 1 corresponde al punto 1, conteniendo los histrogramas alusivos al centro del circulo. Por su parte la seccion 2 contiene las graficas de los parametros del ajuste realizdo por el metodo de montecarlo.

\end{abstract} 

\section{Punto 1}
Las primeras graficas entonces son los histogramas de las coordenadas del centro para los datos de Canalionico.txt y Canalionico1.txt y la sus respectivas graficas con el radio maximo encontrado:

\begin{figure}[!h]
\centering
\includegraphics[scale = 0.4]{histx.png}
\caption{Histograma de la coordenaa x del centro del circulo, canalionico.txt}
\end{figure}

\begin{figure}[!h]
\centering
\includegraphics[scale = 0.4]{histy.png}
\caption{Histograma de la coordenaa y del centro del circulo, canalionico.txt}
\end{figure}

\begin{figure}[!h]
\centering
\includegraphics[scale = 0.4]{canal.png}
\caption{Circulo de radio maximo para el canal ionico.txt}
\end{figure}


\begin{figure}[!h]
\centering
\includegraphics[scale = 0.4]{histx1.png}
\caption{Histograma de la coordenaa x del centro del circulo, canalionico1.txt}
\end{figure}

\begin{figure}[!h]
\centering
\includegraphics[scale = 0.4]{histy1.png}
\caption{Histograma de la coordenaa y del centro del circulo, canalionico1.txt}
\end{figure}

\begin{figure}[!h]
\centering
\includegraphics[scale = 0.4]{canal1.png}
\caption{Circulo de radio maximo para el canal ionico1.txt}
\end{figure}

\section{Punto 2}

Se muestran los histogramas de los parametros ademas de su comportamiento con la funcion de maxima verosimilitud, finalmente mostrando el ajuste realizado:

\begin{figure}[!h]
\centering
\includegraphics[scale = 0.4]{histr.png}
\caption{Histograma del parametro r}
\end{figure}

\begin{figure}[!h]
\centering
\includegraphics[scale = 0.4]{histc.png}
\caption{Histograma del parametro c}
\end{figure}


\begin{figure}[!h]
\centering
\includegraphics[scale = 0.4]{rvsl.png}
\caption{Likelihood en funcion de r}
\end{figure}

\begin{figure}[!h]
\centering
\includegraphics[scale = 0.4]{cvsl.png}
\caption{Likelihood en funcion de c}
\end{figure}


\begin{figure}[!h]
\centering
\includegraphics[scale = 0.7]{ajuste.png}
\caption{Ajuste de la ecuacion junto a los datos observados}
\end{figure}

\end{document}
